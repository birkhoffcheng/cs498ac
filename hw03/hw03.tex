\documentclass[12pt]{article}
\usepackage{mdframed}
\usepackage{fullpage,latexsym,amsthm,latexsym,amssymb}
\usepackage{amstext,amsfonts,amsmath,graphicx}
\usepackage{multicol}
\usepackage{hyperref}
\usepackage{palatino}
\usepackage{amssymb}

\oddsidemargin = -0.5 in
\addtolength{\textwidth}{0.8in}
\addtolength{\textheight}{0.2in}

%% Theorem statements %%
% THEOREMS -------------------------------------------------------
\newtheorem{theorem}{Theorem}[section]
\theoremstyle{definition}
\newtheorem{definition}{Definition}
\numberwithin{equation}{section}
%% MISC DEFINITIONS %%%%%%%%%%%%%%%%%%%%%%%%%%%%%%%%%%%

\newcommand{\qtrinfo}{CS/ECE 498 AC 3/4 Fall 2020}
\newcommand{\remove}[1]{}
\newcommand{\headnote}{
\begin{multicols}{2}\small
\begin{itemize}

\item
Please write your name, UID, \textbf{and the names of anyone you
collaborated with} in the spaces provided and attach this sheet to the
front of your solutions.
\item Please write your answers in a neat and readable hand-writing. \textbf{Each answer should be on a separate page.} There are plenty of extra spaces, you do not need to fill them all. You should expect answers to be fairly short.
\item Always explain your answers. When a proof is requested, you should provide a rigorous proof. This usually involves a reduction.
\item 10\% of the points will be given if your answer is ``I don't know''.
However, if instead of writing ``I don't know'' you write things that do not make any sense, no points will be given.
\item The homework is expected to take several ($\geq 8$) hours. Start early.
\end{itemize}
\end{multicols}
}

\newcommand{\cD}{\mathcal{D}}
\newcommand{\cE}{\mathcal{E}}
\newcommand{\hwhead}[2]{
	\raggedleft{Name: \underline{Zhiqi Cheng} \\ \medskip
		UID: \underline{678676902} \\ \medskip
		Collaborators: \underline{\hspace{3in}} \\ \medskip}
	\bigskip
	\center{\LARGE{\qtrinfo \ -- Problem Set 3}}
	\center{\Large{Due #2}}
	\bigskip

	\raggedright
	\headnote
	\smallskip
}
%%%%%%%%%%%%%%%%%%%%%%%%%%%%%%%%%%%%%%%%%%%%%%%%%%%%%
\addtolength{\topmargin}{-1cm}
\addtolength{\textheight}{2cm}

\begin{document}
\hwhead{2}{Monday, November 9, 6pm (Submit via Gradescope)}

\paragraph{Honor Code (READ CAREFULLY).}
When submitting my solution set, I agree to follow the following rules concerning homework and problem sets:
\begin{itemize}
\item	When formulating a high-level approach to the problems I am given, I am allowed to ask the TA for help, and to collaborate with my fellow UIUC students, and I will acknowledge in writing the names of every student that I collaborated with on my work.
I understand that collaboration and study groups are encouraged. My grade will not suffer if I acknowledge help from other students, but may suffer if I collaborate without explicit acknowledgement.
\item	When writing solutions or code, I will do so entirely on my own, without the help of any other person or any other student’s notes or solutions or code. I will not discuss implementation details with other students.
\item	I will not seek out or make use of any outside source of information, except sources explicitly linked to in the exam. Outside sources include but are not limited to: any information found online (except on the course website or linked in the homework), other textbooks or articles.	If necessary, I may ask for permission in advance from the TA to deviate from this rule.
The textbook for the course (Boneh-Shoup) and my own notes are excepted from this rule.
\item I understand that I may be suspended from UIUC for violating this honor code.
\item	If I have difficulty with understanding lectures or the book due to difficulty with the English language, or if someone asks for my help because of difficulties with language, I will discuss the situation with the professor or TA as soon as possible, to ensure that the honor code is upheld.	I will follow these rules in good faith, and not try to find “loopholes” that violate the spirit of the rules.	If I am unsure about what the spirit of the rule is or what a rule means, then I will ask for clarification from the professor or TA, and I will adhere to rules as clarified.
\end{itemize}

\begin{enumerate}
\newpage
\item
{\bf Verifiable Random Functions (25 points).}
Let $\mathbb{G}$ be a cyclic group of prime order $q$ generated by $g \in \mathbb{G}$.
Let $F$ be a PRF such that $F(k, m) := {H(m)}^k$ for $k \in \mathbb{Z}_q, m \in \mathcal{M}$, where $H: M \rightarrow \mathbb{G}$ is a hash function modeled as a random oracle.
\begin{enumerate}
\item {\bf (5 points).} Prove that $F$ is a PRF under the DDH assumption (when $H$ is modeled as a random oracle).
\item
A verifiable random function (VRF) is a PRF, with the additional property that anyone can verify that the PRF value at a given point is computed correctly. Specifically, a VRF defined over $(X , Y)$ is a triple of efficient algorithms $(G', F', V')$, where algorithm $G'$ outputs a public key $pk$ and a secret key $sk$. Algorithm $F'$ is invoked as $(y, \pi) \leftarrow F'(sk, x)$ where $x \in X , y \in Y$, and where $\pi$ is a validity proof. Algorithm $V'$ is invoked as $V'(pk, x, y, \pi)$, and outputs	accept or reject. We say that $y$ is the value of the VRF at the point $x$, and $\pi$ is the validity proof for $y$. The VRF must satisfy the following properties:
\begin{itemize}
\item {\bf Correctness.} For all $(pk, sk)$ output by $G'$, and all $x \in X$, if $(y, \pi) \leftarrow F'(sk, x)$ then $V'(pk, x, y, \pi) =$ accept.
\item {\bf Uniqueness.}	For all $pk$ and every $x \in X$, only a single $y \in Y$ can have a valid proof $\pi$. More precisely, if $V'(pk, x, y, \pi) = V'(pk, x, y_0, \pi_0) =$ accept then $y = y_0$.
This ensures that even with the secret key, an adversary cannot lie about the value of the VRF at the point $x$.
\item {\bf Security.} VRF security is defined using two experiments, analogous to the characterization of a PRF. In both experiments, the challenger generates $(pk, sk)$ using $G'$, and gives $pk$ to the adversary. The adversary then makes a number of function queries and a single test query (with any number of function queries before and after the test query). In a function query, the adversary submits $x \in X$ and obtains $(y, \pi) \leftarrow F'(sk, x)$.
In the test query, the adversary submits $\widetilde{x} \in X$: in one experiment, he obtains $\widetilde{y}$, where $(\widetilde{y}, \widetilde{\pi}) \leftarrow F'(sk, \widetilde{x})$; in the other experiment, he obtains a random $\widetilde{y} \in Y$. The test point $\widetilde{x}$ is not allowed among the function queries. The VRF is secure if the adversary cannot distinguish the two experiments.
\end{itemize}
We will try to build a VRF using the PRF $F$ discussed in the previous item.
Let $\Pi$ be the Chaum-Pedersen protocol, and let $\Phi$ denote the non-interactive version of this protocol (with negligible soundness error) in the random oracle model, as discussed in class.

Our VRF is $(G', F', V')$, which is defined over $(\mathcal{M}, \mathbb{G})$.
$G'$ chooses $k \in \mathbb{Z}_q$ at random, $k$ is the secret key, and $g^k$ is the public key;
$F'(k, m) := (y, \pi)$, where $y = F(k, m) = {H(m)}^k$ and $\pi$ is a proof, generated using $\Phi$, that $(H(m), g^k, {H(m)}^k)$ is a DH-triple; $V'(g^k, m, y, \pi)$ checks that $(H(m), g^k, y)$
is a DH-triple by verifying the proof $\pi$ using $\Phi$.
\begin{enumerate}
\item {\bf (5 points).} Describe the functions $F'$ and $V'$ in detail.
\item {\bf (10 points).} Using the ZK property for $\Phi$, show that if we model both $H$ and $H'$ as random oracles, then under the DDH assumption for $\mathbb{G}$, the VRF $(G', F', V')$ satisfies the VRF security property defined above.
\item {\bf (5 points).} This VRF does not satisfy the uniqueness property defined above. Nevertheless, it does satisfy a weaker, but still useful property. Using the soundness property for $\Phi$, show that it is infeasible for an adversary to come up with a triple $(m, y, \pi)$ such that $V'(g^k, m, y, \pi) = $ accept, yet $y \neq F(k, m)$.
\end{enumerate}
\end{enumerate}
\small{\em Note: In parts (a), (b) (ii) and (b) (iii), provide explicit reductions.}

\begin{enumerate}
	\item Assume DDH holds. For the sake of contradiction, suppose we have an adversary $A$ who can break $F$ with probability $\epsilon$. Then we can use $A$ to break the DDH assumption as follows:
	\begin{enumerate}
		\item Request a DDH triple $(g^a, g^b, g^c)$ where $c$ might be a random number or $ab$.
		\item Get message $m_1$ from the adversary and program the random oracle such that $H(m_1)=(g^a)^{r_1}$ for some random $r_1$.
		\item Send $(g^c)^{r_1}$ to the adversary.
		\item For any following query $m_i$, continue making $H(m_i)=(g^a)^{r_i}$ or $(g^b)^{r_i}$ and sending $(g^c)^{r_i}$
		\item If the adversary outputs 0, then we output 0 as well (c is a multiple of a and b). If the adversary outputs 1, we also output 1 (c is just random).
	\end{enumerate}
	As such, we successfully break the DDH assumption with probability $\frac{ab\epsilon}{q}$, there is a contradiction. Therefore $F$ is a secure PRF under the DDH assumption.
	$\blacksquare$
	\item
	\begin{enumerate}
		\item We define $F'(k, m) := (y, \pi)$ where $y = F(k, m) = {H(m)}^k$.\\
		Since ${H(m)} = g^a$ for some $a$ because ${H(m)}\in\mathbb{G}$, $(H(m), g^k, {H(m)}^k)$ must be a DH-triple. Model $H'$ as a random oracle. We construct the non-interactive proof $\pi$ as follows:\\
		1. Sample a random $d$ from $\mathbb{Z}_q$ and compute $v = g^d, w = {H(m)}^d$.\\
		2. Compute challenge $c = H'(H(m), g^k, {H(m)}^k, v, w)$.\\
		3. Compute $e = d + kc$.\\
		4. Output $\pi = (v, w, e)$.\\
		$V'(g^k, m, y, \pi)$ where $\pi = (v, w, e)$ first computes the challenge $c = H'(H(m), g^k, {H(m)}^k, v, w)$ and check if $g^e = v \cdot (g^k)^c$.\\
		Then, it would then check ${H(m)}^e = w \cdot {H(m)}^c$.\\
		If both checks pass, it outputs accept, otherwise it outputs reject.
		\item Assume the DDH assumption holds for $\mathbb{G}$. For the sake of contradiction, suppose there is an adversary $A$ that is able to distinguish $(\widetilde{y}, \widetilde{\pi})$ output by $F'$ from $(\widetilde{y}, \widetilde{\pi})$ where $y$ is random. Then, we can use $A$ to break the DDH assumption as follows:\\
		1. Request a DDH triple $(g^a, g^b, g^z)$ where $z$ might be a random number or $ab$.\\
		2. Set public key $g^k = g^b$.\\
		3. For a signing query $x$, program $H$ to make $H(x) = g^\alpha$ for some $\alpha$ and output $y = (g^\alpha, g^k, (g^k)^\alpha)$.\\
		4. Program $H'$ to output the correct $c$ so that verification passes.\\
		5. When $A$ sends $m$ and requests the challenge, program $H$ to make $H(m) = g^a$ and send $g^z$.\\
		6. If $A$ outputs 0, we output 0; if $A$ outputs 1, we output 1.\\
		As such, we successfully break the DDH assumption, there is a contradiction. Therefore, the VRF satisfies the security property defined above.
		$\blacksquare$
		\item Suppose discrete logarithm is hard. For the sake of contradiction, assume there is an efficient adversary $A$ that can produce a triple $(m, y, \pi)$ such that $V'(g^k, m, y, \pi) = $ accept, yet $y \neq F(k, m)$. We can use $A$ to compute the discrete log problem as follows:\\
		1. Request a discrete log problem $g^x$ and set the public key $g^k = g^x$.\\
		2. Let $A$ make as many queries as it likes until it decides to forge a signature.\\
		3. Split the game into two. In one game, let $H'(H(m), g^k, {H(m)}^k, v, w) = h_1$ be the original hash. In the other game, program $H'$ to make $H'(H(m), g^k, {H(m)}^k, v, w) = h_2$.\\
		4. With the two signatures $e_1 = d + kh_1, e_2 = d + kh_2$, we can compute $x = k = \frac{e_1 - e_2}{h_1 - h_2}$.\\
		As such, we have solved the discrete logarithm in polynomial time. There is a contradiction. Therefore, such adversary must not exist.
		$\blacksquare$
	\end{enumerate}
\end{enumerate}

\newpage
\item
{\bf Schnorr Signatures with Related Keys (15 points.)} Let $vk = g^x \in \mathbb{G}$ be a random verification key for the non-interactive Schnorr signature/identification scheme, as discussed in the lecture.
Define $n$ related verification keys $vk_1, \ldots, vk_n$ by setting $vk_i = (vk) \cdot g^i \in \mathbb{G}$ for $i = \{1, 2, \ldots, n\}$.
\begin{enumerate}
\item {\bf (5 points).} Show that the scheme we saw in class is {\em insecure} w.r.t. related verification keys as described above. In particular, show that an adversary that is given $vk_1, \ldots, vk_n$ generated as above, and a signature on message $m$ with respect to $vk_j$ for some $j \in [n]$ will be able to construct a signature on $m$ w.r.t. a different $vk_i$ for some $i \neq j$.
\item {\bf (10 points).} Now suppose we modify the identification scheme so that for any verification key $vk_i$, instead of computing challenge $h = H(m, R)$, $h$ is computed as $H(m,R,vk_i)$.
Show that this scheme {\em is multi-key secure} with respect to the $n$ verification keys $vk_1, \ldots, vk_n$.

That is, consider an adversary that is given $vk$ and can issue signing queries to all $n$ verification keys. Each query is a pair $(vk^{(j)}, m_j )$, for $j = 1, \ldots, Q$, and the response is a signature on $m_j$ with respect to $vk^{(j)} \in \{vk_1, \ldots, vk_n\}$.
Show that the adversary cannot produce, with non-negligible probability, an existential forgery
$(vk, m, \sigma)$ where $vk \in \{vk_1, \ldots, vk_n\}$ and $(vk, m)$ is {\em not} one of the signing queries. Your security argument should be based on the hardness of computing discrete-log in $\mathbb{G}$, where the hash function used in the scheme is modeled as a random oracle.
\end{enumerate}

Answer:
\begin{enumerate}
	\item Given a message $m$ with a signature under verification key $vk_j$ for some $j\in [n]$, an adversary can do the following to forge a signature of $m$ under $vk_i$ for an arbitrary $i\in [n] \land i\neq j$.
	\begin{enumerate}
		\item Get a signature $(R, s_j)$ of $m$ under $vk_j$.
		\item Compute $h = H(m, R)$.
		\item Output signature $(R, s_i)$ under $vk_i$ where $s_i = s_j + h(i-j)$
	\end{enumerate}
	Because $vk$ corresponds to signing key $x$, $vk_i=(vk)\cdot g^i$ corresponds to signing key $x+i$ and $vk_j$ corresponds to signing key $x+j$. The signature of $m$ under $vk_j$ is $s_j = r + h(x+j)$ for some random $r$ and the signature of $m$ under $vk_i$ is $s_i = r + h(x+i)$. Therefore, we can simply add the difference to $s_j$ in order to produce $s_i$ without the knowledge of $x$. Therefore, the scheme we saw in class is insecure w.r.t. related verification keys.
	$\blacksquare$

	\item Assume discrete logarithm is hard. For the sake of contradiction, suppose there is an efficient adversary $A$ that breaks $F$ with the modified hashing scheme with probability $\epsilon$. Then we can break the discrete log assumption with $A$ as follows:
	\begin{enumerate}
		\item Let $A$ interact with the signer until $A$ decides to produce a forgery.
		\item When $A$ has decided to forge a signature on $m$ under $vk_i$, fork the game into two. In one game, when the adversary queries the random oracle for $H(m, R, vk_i)$, program the random oracle to output $h_1$; in the other game, program the random oracle to output $h_2$.\\
		Note that $A$ has to query the random oracle on $(vk_i, m)$ since this message-key pair is not allowed in signing queries.
		\item When the attacker outputs $s_1 = r + h_1(x+i)$ and $s_2 = r + h_2(x+i)$ in their respective games, we know $s_1, s_2, h_1, h_2, i$, so we solve the following linear equation to compute the value of $x$.\\
		$$
		\begin{cases}
			s_1 = r + h_1(x+i)\\
			s_2 = r + h_2(x+i)
		\end{cases}
		$$
	\end{enumerate}
	Therefore, we can solve the discrete logarithm in polynomial time with probability $\epsilon$; there is a contradiction. As such, the new scheme is multi-key secure.
	$\blacksquare$
\end{enumerate}

\newpage
\item {\bf Revisiting conference key setup/multi-party key exchange (20 points).}
Parties $A_1,A_2$ and $A_3$ wish to generate a secret conference key.
All parties should know the conference key, but an eavesdropper should not be able to output the key. They decide to use bilinear maps: there is a public generator $g$ of a group $\mathbb{G}_1$ and $q = |\mathbb{G}_1|$. Suppose there exists a bilinear map $e: \mathbb{G}_1 \times \mathbb{G}_1 \rightarrow \mathbb{G}_T$ such that $e(g^a, g^b) = {e(g,g)}^{ab}$, where $a, b \in \{0,1,\ldots, q-1\}$ and $e(g,g)$ is a generator of $\mathbb{G}_T$.

\begin{enumerate}
\item {\bf (8 points).} Consider the following two computational assumptions:
\begin{itemize}
\item {\bf Assumption 1:} The distributions $\cD_0$ and $\cD_1$ are indistinguishable where
\begin{itemize}
\item $\cD_0 = (g, g^a, g^b, g^c, g^{abc})$ where $a, b, c \leftarrow \{0, 1, \ldots, q-1\}$.
\item $\cD_1 = (g, g^a, g^b, g^c, g^{d})$ where $a, b, c, d \leftarrow \{0, 1, \ldots, q-1\}$.
\end{itemize}
\item {\bf Assumption 2:} The distributions $\cE_0$ and $\cE_1$ are indistinguishable where
\begin{itemize}
\item $\cE_0 = (g, g^a, g^b, g^c, {e(g,g)}^{abc})$ where $a, b, c \leftarrow \{0, 1, \ldots, q-1\}$.
\item $\cE_1 = (g, g^a, g^b, g^c, {e(g,g)}^{d})$ where $a, b, c, d \leftarrow \{0, 1, \ldots, q-1\}$.
\end{itemize}
\end{itemize}
Show that Assumption 1 implies Assumption 2.

\item {\bf (12 points).}
Devise a {\em non-interactive} key exchange scheme, where parties $(A_1, A_2, A_3)$ need only simultaneously broadcast a single message to each other. After this point, $(A_1, A_2, A_3)$ can derive a shared secret key that is known only to these three parties.

Under Assumption 1, prove that an eavesdropper that observes the messages sent by $(A_1, A_2, A_3)$ cannot guess the secret key. Namely, show that if there exists an efficient algorithm $A$ that given the messages sent by $(A_1, A_2, A_3)$ outputs the shared secret key, then there also exists an efficient algorithm $B$ that breaks Assumption 1.
\end{enumerate}

\newpage
\begin{enumerate}
	\item Suppose Assumption 1 holds. For the sake of contradiction, suppose there is an adversary $A$ that can break Assumption 2, i.e. $A$ can distinguish $\cE_0$ from $\cE_1$. Then we can use $A$ to distinguish $\cD_0$ from $\cD_1$ as follows:
	\begin{enumerate}
		\item Request $\cD_i = (g, g^a, g^b, g^c, g^x)$ where $x$ could be $abc$ or a random $d$.
		\item Compute $\cE_i = (g, g^a, g^b, g^c, e(g^x,g))$ from $\cD_i$ and send to $A$.
		\item If $A$ outputs $i = 0$, we output $i = 0$; if $A$ outputs $i = 1$, we also output $i = 1$.
	\end{enumerate}
	Because $\forall a, b, e(g^a, g^b) = {e(g,g)}^{ab}$, so $e(g^x,g) = {e(g,g)}^x$. Therefore, since $A$ can distinguish ${e(g,g)}^{abc}$ from ${e(g,g)}^d$, we can distinguish $g^{abc}$ from $g^d$. There is a contradiction, so Assumption 1 must imply Assumption 2.
	$\blacksquare$
	\item We can have the key exchange scheme as follows:
	\begin{enumerate}
		\item Parties $(A_1, A_2, A_3)$ generate secret $a, b, c$ and broadcast $g^a, g^b, g^c$ respectively.
		\item Each party then compute $e$ over the other two parties' messages raised to their own secret, then they will all come up with the shared secret ${e(g, g)}^{abc}$\\
		For example, party $A_1$ can compute the shared secret ${e(g^b, g^c)}^a = {e(g,g)}^{abc}$
	\end{enumerate}
	Because under Assumption 1, which implies Assumption 2, no eavesdropper will be able to distinguish ${e(g,g)}^{abc}$ from ${e(g,g)}^d$ for a random $d$. Therefore, they will not be able to come up with the shared secret $s$.\\
	For the sake of contradiction, suppose there is an efficient adversary $A$ that given $(g, g^a, g^b, g^c)$ can output the shared secret ${e(g, g)}^{abc}$. Then we break Assumption 2 with the following algorithm $B$:
	\begin{enumerate}
		\item Request $\cE_i = (g, g^a, g^b, g^c, {e(g,g)}^x)$ and send $(g, g^a, g^b, g^c)$ to $A$.
		\item Compare the output of $A$ to the last field of $\cE_i$. If the two match, we output $i = 0$, otherwise we output $i = 1$.
	\end{enumerate}
	As such, we successfully break Assumption 2, which breaks Assumption 1 with the method described in (a). There is a contradiction, so under Assumption 1 an eavesdropper that observes the messages sent by $(A_1, A_2, A_3)$ cannot guess the secret key.
	$\blacksquare$
\end{enumerate}
\end{enumerate}

\vfill

\paragraph{Feedback.} Did you find this homework assignment too easy/too hard/just right? How is the pace of the course so far? Please add any feedback about lectures and/or homework that would help improve the course.
\newline\newline
Problem 1b too hard, others are just right.
\end{document}
